\documentclass[a4paper,12pt]{book} 
\usepackage{graphicx}




\begin{document} 
	
\title{Requirement Analysis and Specification Document RASD}
\author{KONG XIANGYI\and ZHANG YUEDONG}
\date{\today}


\frontmatter
\maketitle


\tableofcontents

\mainmatter

\chapter{Introduction} \label{introduction}

\section{Purpose}
This document is Requirement Analysis and Specification Document(RASD). The main purpose of this document is the following points
\begin{itemize}
	\item Communicates an understanding of the requirements to the audience and explains both the application domain and the system to be developed.
	\item Contractual: Make this project formal and written so that it has legal effect.
	\item As the baseline for project planning and estimation. i.e. size, cost, schedule. 
	\item As the baseline for software evaluation
		\subitem It can support system testing, verification and validation activities
		\subitem It should contain enough information to verify whether the delivered system meets requirements
	\item As the baseline for change control, such as requirements change, software evolves.
\end{itemize}
And this RASD has the following intended audiences
\begin{itemize}
	\item Costumers \& Users : Some user may interest in validating system goals and high-level description of functionalities.
	\item Systems and Requirements Analysts: The RASD may help them to write various specifications of other systems that inter-relate.
	\item Developers, Programmers: The RASD may help the to implement the requirements
	\item Testers: The RASD may help the to determine that the requirements have been met
	\item Project Managers: The RASD may help them to measure and control the analysis and development processes
\end{itemize}

\section{Scope}
\subsection{Description of the given problem}

At the end of 2019, a global epidemic broke out and swept almost all countries in the world in just a few months. Starting in 2020, people's life rhythm has been completely disrupted by this epidemic, a lot of cities are blocked, people are allowed to exit their homes only for essential needs, everyone had to wear masks and respect the social-distancing at least 1.5 m. In the public area, the human community has to take measures to avoid the crazy spread of the virus. Restaurants began to use dividers to separate the table, supermarkets and museums began to restrict flow of people, the school also adopted into two classes mode: online and onsite.

In this situation, a new problem arises, how to delay the spread of the virus through technical means? 

Since grocery shopping is the most needed activity under the lock-down, so let’s narrow the problem to grocery shopping.

In the supermarket, In order to meet these strict rules, many challenges have arisen, so, we can turn to technology, in particular to software applications, to help navigate the challenges created by the imposed restrictions.

So, this project appeared - Customers Line-up(CLup).


Customers Line-up(CLup) is an user-friendly application, it has two main goals. 
\begin{itemize}
	\item  First of all, the CLup have to allow store managers to regulate the influx of people in the building.
	\item  And then, it will help people to avoid lining outside of stores for hours.	
\end{itemize}

\subsection{World Phenomena}
\begin{center}
	\begin{tabular}{ c|c } 
		\hline
		$WP_1$ & cell \\ 
		\hline
		$WP_2$ & cell \\ 
		\hline
		$WP_3$ & cell \\ 
		\hline
	\end{tabular}
\end{center}

\subsection{Shared Phenomena}
\begin{center}
	\begin{tabular}{ c|c } 
		\hline
		$SP_1$ & cell \\ 
		\hline
		$SP_2$ & cell \\ 
		\hline
		$SP_3$ & cell \\ 
		\hline
	\end{tabular}
\end{center}

\subsection{Goals}
\begin{center}
	\begin{tabular}{ c|c } 
		\hline
		$G_1$ & cell \\ 
		\hline
		$G_2$ & cell \\ 
		\hline
		$G_3$ & cell \\ 
		\hline
	\end{tabular}
\end{center}


\section{Definitions, acronyms, abbreviations}

\subsection{Definitions}
\begin{itemize}
	\item Customer:
	\item QR code:
	\item Store Manager:
\end{itemize}


\subsection{Acronyms}

\begin{itemize}
	\item RASD – Requirement Analysis and Specification Document
	\item CLup - Customers Line-up
	\item UI - User Interface
\end{itemize}

\subsection{Abbreviations}

\begin{itemize}
	\item  $WP_n$ : n-th world phenomena
	\item  $SP_n$ : n-th shared phenomena
	\item  $G_n$ : n-th goal
	\item  $D_n$ : n-th domain assumption
	\item  $R_n$ : n-th functional requirement
\end{itemize}


\section{Reference documents}

\begin{itemize}
	\item Specification Document: "R\&DD Assignment A.Y. 2020-2021"
	\item Slides of the "Software Engineering 2" course A.Y. 2020-2021
	\item IEEE Recommended Practice for Software Requirements Specifications - IEEE Std 830-1998
\end{itemize}
	
\section{Overview}
The RASD document consists of five chapters.

\textbf{Chapter \ref{introduction}} is the introduction chapter, it's an overview of the RASD and project, 
it describes the purpose of the CLup.

\textbf{Chapter \ref{OverallDescription}} 

\textbf{Chapter \ref{SpecificRequirements}} 

















\chapter{Overall Description} \label{OverallDescription}
\section{Product perspective}
\section{Product functions}
\section{User characteristics}
\section{Constraints}
\section{Assumptions and Dependencies}

\chapter{Specific Requirements} \label{SpecificRequirements}
\section{External Interface Requirements}
\subsection{User Interfaces}
\subsection{Hardware Interfaces}
\subsection{Software  Interfaces}
\subsection{Communication Interfaces}

\newpage
\section{Functional Requirements}
\subsection{User Class 1}
\subsubsection{Functional Requirement 1.1}
\subsection{User Class 2}
\subsubsection{Functional Requirement 2.1}

\newpage
\section{Performance Requirements}

\newpage
\section{Design Constraints}
\subsection{Standards compliance}
\subsection{Hardware limitations}

\newpage
\section{Software System Attributes}
\subsection{Reliability}
\subsection{Availability}
\subsection{Security}
\subsection{Maintainability}
\subsection{Portability}

\newpage
\section{Other Requirements}


\backmatter


\end{document}
